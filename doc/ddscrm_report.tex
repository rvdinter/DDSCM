%%
%% Copyright 2007, 2008, 2009 Elsevier Ltd
%%
%% This file is part of the 'Elsarticle Bundle'.
%% ---------------------------------------------
%%
%% It may be distributed under the conditions of the LaTeX Project Public
%% License, either version 1.2 of this license or (at your option) any
%% later version.  The latest version of this license is in
%%    http://www.latex-project.org/lppl.txt
%% and version 1.2 or later is part of all distributions of LaTeX
%% version 1999/12/01 or later.
%%
%% The list of all files belonging to the 'Elsarticle Bundle' is
%% given in the file `manifest.txt'.
%%

%% Template article for Elsevier's document class `elsarticle'
%% with harvard style bibliographic references
%% SP 2008/03/01
%%
%%
%%
%% $Id: ddscrm_report.tex 4 2009-10-24 08:22:58Z rishi $
%%
%%
%%\documentclass[preprint,authoryear,12pt]{elsarticle}

%% Use the option review to obtain double line spacing
%% \documentclass[authoryear,preprint,review,12pt]{elsarticle}

%% Use the options 1p,twocolumn; 3p; 3p,twocolumn; 5p; or 5p,twocolumn
%% for a journal layout:
%%\documentclass[final,authoryear,1p,times]{elsarticle}
%\documentclass[final,authoryear,1p,times,twocolumn]{elsarticle}
%% \documentclass[final,authoryear,3p,times]{elsarticle}
%% \documentclass[final,authoryear,3p,times,twocolumn]{elsarticle}
%% \documentclass[final,authoryear,5p,times]{elsarticle}
\documentclass[final,authoryear,5p,times,twocolumn, 12pt]{elsarticle}

%% if you use PostScript figures in your article
%% use the graphics package for simple commands
%% \usepackage{graphics}
%% or use the graphicx package for more complicated commands
%% \usepackage{graphicx}
%% or use the epsfig package if you prefer to use the old commands
%% \usepackage{epsfig}

%% The amssymb package provides various useful mathematical symbols
\usepackage{amssymb}
\usepackage{enumitem}
\usepackage{tabularx}
\usepackage{multicol}
%% The amsthm package provides extended theorem environments
%% \usepackage{amsthm}

%% The lineno packages adds line numbers. Start line numbering with
%% \begin{linenumbers}, end it with \end{linenumbers}. Or switch it on
%% for the whole article with \linenumbers after \end{frontmatter}.
%% \usepackage{lineno}

%% natbib.sty is loaded by default. However, natbib options can be
%% provided with \biboptions{...} command. Following options are
%% valid:

%%   round  -  round parentheses are used (default)
%%   square -  square brackets are used   [option]
%%   curly  -  curly braces are used      {option}
%%   angle  -  angle brackets are used    <option>
%%   semicolon  -  multiple citations separated by semi-colon (default)
%%   colon  - same as semicolon, an earlier confusion
%%   comma  -  separated by comma
%%   authoryear - selects author-year citations (default)
%%   numbers-  selects numerical citations
%%   super  -  numerical citations as superscripts
%%   sort   -  sorts multiple citations according to order in ref. list
%%   sort&compress   -  like sort, but also compresses numerical citations
%%   compress - compresses without sorting
%%   longnamesfirst  -  makes first citation full author list
%%
%% \biboptions{longnamesfirst,comma}

% \biboptions{}

%\journal{Nuclear Physics B}
\makeatletter
\def\ps@pprintTitle{%
 \let\@oddhead\@empty
 \let\@evenhead\@empty
 \def\@oddfoot{}%
 \let\@evenfoot\@oddfoot}
\makeatother
\begin{document}

\begin{frontmatter}

%% Title, authors and addresses

%% use the tnoteref command within \title for footnotes;
%% use the tnotetext command for the associated footnote;
%% use the fnref command within \author or \address for footnotes;
%% use the fntext command for the associated footnote;
%% use the corref command within \author for corresponding author footnotes;
%% use the cortext command for the associated footnote;
%% use the ead command for the email address,
%% and the form \ead[url] for the home page:
%%
%% \title{Title\tnoteref{label1}}
%% \tnotetext[label1]{}
%% \author{Name\corref{cor1}\fnref{label2}}
%% \ead{email address}
%% \ead[url]{home page}
%% \fntext[label2]{}
%% \cortext[cor1]{}
%% \address{Address\fnref{label3}}
%% \fntext[label3]{}

\title{Forecasting Demand of Perishable Products Using Machine Learning}

%% use optional labels to link authors explicitly to addresses:
%% \author[label1,label2]{<author name>}
%% \address[label1]{<address>}
%% \address[label2]{<address>}

\author{Raymon van Dinter}
%\ead{r.dinter.rvd@gmail.com}
\address{ORL-33806\\Data Driven Supply Chain Management\\Wageningen University and Research}

\begin{abstract}
%% Text of abstract
\end{abstract}

\begin{keyword}
%% keywords here, in the form: keyword \sep keyword
Data Driven Supply Chain Management \sep Machine Learning \sep Deep Learning \sep Regression
%% MSC codes here, in the form: \MSC code \sep code
%% or \MSC[2008] code \sep code (2000 is the default)

\end{keyword}

\end{frontmatter}

% \linenumbers

%% main text
\section{Introduction}
\label{sec:introduction}
Retailers that sell perishable products should manage their supply chain in order to have the least amount or waste or shortages as possible. A retailer in barbecue products was selected to be further investigated. This retailer recorded its demand on perishable products for 4 years. A model was created to simulate the operation of common retailers handling with perishable products. An attempt to improve usual demand prediction was sought after using machine learning models, fit with features of weather data from the Royal Dutch Meteorological Institute (KNMI), and with demand targets.

\section{Retailer simulation}
In order to investigate the operation of a common retailer of perishable products, a quantitative model was developed. This model kept in mind a daily routine:
\begin{itemize}[]
	\item At the start of every day, the order up to level \textit{S} will be set by predicting demand \textit{D}. There are various methods to calculate \textit{S}, which will be explained later in this section.
	\item Order quantity \textit{Q} is determined by subtracting the daily starting inventory \textit{I} from \textit{S}.
	\item During the day, customers buy products. This demand is subtracted FIFO-wise from \textit{I}. If demand is higher than stock, a shortage is counted.
	\item Next day starting inventory \textit{I} is determined by adding \textit{Q} to \textit{I} FIFO-wise.
	\item Waste is computed by counting the expired products.
	\item At the end of the iteration, average shortage and waste are computed.
\end{itemize}
\subsection{Determining D}
In this simulation model, two methods of determining demand \textit{D} were implemented. The first method was simulating \textit{D}, the second was using a manual input (a \texttt{list}), which could be a predicted \textit{D} using a machine learning model.

\subsection{Computation of S}
Order up to level \textit{S} was computed by using the formula: 
\begin{equation}
S = \mu_{R+L} + z(\sigma_{R+L})
\end{equation}
Where \textit{R} means the daily ordering and \textit{L} stands for the lead time. $\mu_{R+L}$ is the demand over $R+L$ days, where $z(\sigma_{R+L})$ is a safety stock that is added to the expected demand over review period plus lead time. 
\section{Prediction models}
\label{sec:models}
All demand prediction models are elaborated in this section.
\subsection{Analysis and Choice}
%TODO Analysis and choice of prediction models
In order to forecast demand, several prediction models were created. These models included:
\begin{itemize}[noitemsep]
 \item Constant demand
 \item Mean average
 \item Linear regression
 \item Decision tree regression
 \item Support vector regression
 \item Random forest regression
 \item Gradient boosting regression
 \item Multi-layer perceptron regression
\end{itemize}
\subsection{Parameter settings}
%TODO Prediction model and param settings
In order to efficiently evaluate each machine learning model, the \texttt{GridSearchCV} object from the \texttt{sklearn} library has been used. Parameters for each of these models have been shown in
\begin{table}[b]
\begin{tabular}{p{0.5\columnwidth}p{0.5\columnwidth}}
\hline
Model                             & Parameter values                                                                                                  \\ \hline
Linear regression                 & \{'normalize': False\}                                                                                            \\
Decision tree regression          & \{'max\_depth': 13, 'random\_state': 0\}                                                                          \\
Support vector regression         & \{'C': 100, 'gamma': 0.01\}                                                                                       \\
Random forest regression          & \{'max\_depth': 8, 'max\_features': 'auto', 'n\_estimators': 500, 'random\_state': 0\}                            \\
Gradient boosting regression      & \{'learning\_rate': 0.1, 'max\_depth': 5, 'random\_state': 0\}                                                    \\
Multi-layer perceptron regression & \{'alpha': 1e-06, 'hidden\_layer\_sizes': {[}10{]}, 'max\_iter': 1000000, 'solver': 'lbfgs, 'random\_state': 0'\} \\ \hline
\end{tabular}
\caption{Model Parameters}
\label{tab:params}
\end{table}

\subsection{Quantitative comparison}
%TODO Quantitative comparison of the models

\subsection{Impact of including weather info}
%TODO Discuss impact of including weather info

\section{Results}
\label{sec:results}
%TODO Model reports on RMSE on train and test set
%TODO report relative fill and waste rate

\section{Conclusions and Recommendations}
\label{sec:conclusion}



%% The Appendices part is started with the command \appendix;
%% appendix sections are then done as normal sections
%% \appendix

%% \section{}
%% \label{}

%% References
%%
%% Following citation commands can be used in the body text:
%%
%%  \citet{key}  ==>>  Jones et al. (1990)
%%  \citep{key}  ==>>  (Jones et al., 1990)
%%
%% Multiple citations as normal:
%% \citep{key1,key2}         ==>> (Jones et al., 1990; Smith, 1989)
%%                            or  (Jones et al., 1990, 1991)
%%                            or  (Jones et al., 1990a,b)
%% \cite{key} is the equivalent of \citet{key} in author-year mode
%%
%% Full author lists may be forced with \citet* or \citep*, e.g.
%%   \citep*{key}            ==>> (Jones, Baker, and Williams, 1990)
%%
%% Optional notes as:
%%   \citep[chap. 2]{key}    ==>> (Jones et al., 1990, chap. 2)
%%   \citep[e.g.,][]{key}    ==>> (e.g., Jones et al., 1990)
%%   \citep[see][pg. 34]{key}==>> (see Jones et al., 1990, pg. 34)
%%  (Note: in standard LaTeX, only one note is allowed, after the ref.
%%   Here, one note is like the standard, two make pre- and post-notes.)
%%
%%   \citealt{key}          ==>> Jones et al. 1990
%%   \citealt*{key}         ==>> Jones, Baker, and Williams 1990
%%   \citealp{key}          ==>> Jones et al., 1990
%%   \citealp*{key}         ==>> Jones, Baker, and Williams, 1990
%%
%% Additional citation possibilities
%%   \citeauthor{key}       ==>> Jones et al.
%%   \citeauthor*{key}      ==>> Jones, Baker, and Williams
%%   \citeyear{key}         ==>> 1990
%%   \citeyearpar{key}      ==>> (1990)
%%   \citetext{priv. comm.} ==>> (priv. comm.)
%%   \citenum{key}          ==>> 11 [non-superscripted]
%% Note: full author lists depends on whether the bib style supports them;
%%       if not, the abbreviated list is printed even when full requested.
%%
%% For names like della Robbia at the start of a sentence, use
%%   \Citet{dRob98}         ==>> Della Robbia (1998)
%%   \Citep{dRob98}         ==>> (Della Robbia, 1998)
%%   \Citeauthor{dRob98}    ==>> Della Robbia


%% References with bibTeX database:

%\bibliographystyle{elsarticle-harv}
%\bibliography{<your-bib-database>}

%% Authors are advised to submit their bibtex database files. They are
%% requested to list a bibtex style file in the manuscript if they do
%% not want to use elsarticle-harv.bst.

%% References without bibTeX database:

% \begin{thebibliography}{00}

%% \bibitem must have one of the following forms:
%%   \bibitem[Jones et al.(1990)]{key}...
%%   \bibitem[Jones et al.(1990)Jones, Baker, and Williams]{key}...
%%   \bibitem[Jones et al., 1990]{key}...
%%   \bibitem[\protect\citeauthoryear{Jones, Baker, and Williams}{Jones
%%       et al.}{1990}]{key}...
%%   \bibitem[\protect\citeauthoryear{Jones et al.}{1990}]{key}...
%%   \bibitem[\protect\astroncite{Jones et al.}{1990}]{key}...
%%   \bibitem[\protect\citename{Jones et al., }1990]{key}...
%%   \harvarditem[Jones et al.]{Jones, Baker, and Williams}{1990}{key}...
%%

% \bibitem[ ()]{}

% \end{thebibliography}

\end{document}

%%
%% End of file `ddscrm_report.tex'.
